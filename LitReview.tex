\documentclass[a4paper]{article}
\usepackage{graphicx}
\bibliographystyle{alpha}
\usepackage{url}
\usepackage{hyperref}
\usepackage[parfill]{parskip}
\usepackage{amsmath}
\usepackage{fancyhdr}
\usepackage{proof}
\pagestyle{fancy}
\fancyhf{}
\rhead{James Farthing}
\lhead{\leftmark}
\rfoot{Page \thepage}
\title{Literature Review: Using Gamification to create an educational tool based on Natural Deduction Logic }
\author{James Farthing}
\date{20 November 2015}
\begin{document}
\maketitle
\newpage
\tableofcontents
\newpage
\section{Project summary}

The idea behind this project is to take the Natural Deduction logic in mathematics and try to make it easier for older secondary school and university students to understand. In my opinion, an effective way of doing this is through gamification (making a game of it) which will hopefully engage students in a new and different way. As education is not usually presented in this way, research will be undertaken to see how gamifying educational concepts can be helpful for the user. Whilst primarily a software project, where a game will be created, research needs to done to find the most effective way of communicating via games. Research also needs to be done to fully understand the topic of natural deduction, so the concepts can be presented correctly in the game.

\section{Gamification}

Gamification is the idea of making the making a task or situation use game mechanics that does not naturally use this context. It is theoretically possible to make any task into a game into using game-like concepts and this is why gamification has become business in a wide variety of topic areas \cite{muntean2011raising}.

It is important to understand the characteristics of a game to fully understand how a task can take on game like properties. A game is defined by a system in which players engage in abstract challenge, defined by rules, interactivity, and feedback, that results in a quantifiable outcome often eliciting an emotional reaction. \cite{koster2013theory} Gamification of this in my educational context would be to set rules for a user to follow, challenge them with different Natural Deduction concepts and give them constructive, instant feedback in the task which is not usually presented in this format.

\subsection{Uses}

Gamification has been used in a variety of different situations in a number of industries and environments. Whilst it is important to focus on how gamification has been adapted and used for educational purposes, it has to be noted which other areas have used gamification to their advantage and what key concepts they have used to make gaming a successful platform for their product or service.  

Marketing by companies has been a successful way of using gamification \cite{zichermann2011gamification}. They have been able to engage users in new ways by providing apps and websites with games and activities on and consequently consumers are spending more time around that brand. It is important to harness this way of being able to interest people using gaming principles when trying to teach my educational concepts. I think that people can enjoy and learn more effectively if they are engaged in the thing they are doing. Games are a good way of achieving this.

Industry has also used gamification. Studies have investigated whether gaming concepts can increase motivation and morale in the workforce, leading to higher productivity. An article in Frontiers in psychology was positive stated in conclusion that there's "emerging base of evidence that suggests gamification as a promising strategy for promoting loyalty, productivity, and wellbeing in the workplace" \cite{oprescu2014play}.

Whilst education and industry are two different things, there are qualities and skills that are required in both areas. Increasing motivation to learn a particular mathematical formula, for example, could be very beneficial for some students. A higher rate of productivity would also be advantageous, as students often have many tasks ongoing at any one time. An increase in these skills would overall increase the enjoyment of learning and reinforce understanding so students can get a better handle on the key ideas. 

Gamification has also been embraced by the educational sector. This is by far the most important sector for this project. It is important to understand how gamification has helped improve education and helped student to learn. This is the core to the project, that gamification of educational topics will aid students to learn the topics in an easier way. Above all else, making people being able to learn the Natural Deduction logic in an easier way is the main goal of this project.

\subsection{Advantages}

There is an abundance of educational games on the Internet. Educational games have many benefits that will help students learn. Instant feedback, social aspects, competitiveness, games being fun and giving the user hints and tips are all advantages that make games a better way of learning that other formats. These will all be explained in detail below. 

%Instant Feedback
\subsubsection{Instant Feedback}
Games offer instant feedback; the user can immediately know when they have done something right or wrong. This is a big advantage with games, it can offer advice and tips much quicker than a teacher can. My game needs to use this feature to its advantage, because giving someone instant feedback can be rewarding and be much more helpful than waiting weeks for feedback, for example, after handing in a question sheet. 

Research has been undertaken to examine whether giving students instant feedback has improved learning performance \cite{wu2012innovative}. This study, which was published in the British Journal of Educational Technology, found that the enhanced way of presenting their concept maps which gave the students instant feedback "significantly improved the learning achievements of the students". They recognised that giving instant feedback meant that the topic was fresh in the students' minds and therefore could organise, collate and adapt to the information that was given in the feedback.

The paper also suggested that students which used the new system "gave significantly higher ratings" than those who used the standard approach. This questionnaire was asking about learning attitudes towards the course that they were undertaking. From this information, it can be deduced that the students who were questioned seemed to be enjoying their course more due to the enhanced system that was introduced in this study. It is important to recognise that instant feedback was not the only improvement to the enhanced system they were using in the study, but this was a key part of the new system, so must have contributed at least partly to the students' improved learning performance and improved learning attitudes.

Instant feedback can be key motivation for the project, giving that it can improve both student enjoyment and performance. As mentioned by Dempsey et al, instant feedback can more effective than delayed feedback when using "classroom quizzes and (other) materials" \cite{dempsey1993text}. He goes on to conclude that "immediate feedback should be prescribed unless the feedback is delayed systematically for a specialised purpose". 

It is crucial to bear this in mind when creating the game that giving a user instant feedback can be a powerful tool. Because of this, mathematical notation needs to be included so that students can start to learn the Natural Deduction logic concepts. If the game was too abstract, with no relation to the mathematical representation, part of the learning process would be lost.

One way of offering this constructive feedback would be to offer a scoring system. You could get a certain number of points for getting parts of a question correct and bonus points for flawless levels completed. This would instantly tell the player of the game whether they were doing well or not. Players could repeat plays of the game to get higher scores which would increase understanding. This would be because doing a repetitive action multiple times reinforces your learning.

%Social & Competitive
\subsubsection{Social and Competitive Aspects}
Another advantage to Gamification is the ability to offer a social or competitive aspect of your subject matter. This could be through online leaderboards displaying top scores, multiplayer aspects or social media connectivity. Keeping fellow users of the game connected in some way can be important for increasing enjoyment of gameplay and also the amount of time they spend playing the game.

In the current age where 90\% of 16-24 year olds in the UK have smart phones, and being used for at least two hours a day, it can be said that people find that communicating via online methods important \cite{2015Ofcom}. In the same article, it is said that "around half (49\%) of young people aged 18-24 check their phones within five minutes of waking up". This just backs up my point that social interaction and social media via online devices is in the forefront of people's minds throughout their lives. By incorporating this into my game, by offering some kind of social aspect, people will enjoy it more. 

A study undertaken which has looked into how to create engaging games in the health care sector discovered that communication was important for 13 to 16 year olds, which was their target market \cite{Suhonen:2008:SFE}. It also discovered that being in touch with others and interacting with them while playing games was also important to them. The report states that "The company of friends motivates to play games and makes gaming fun". These are two massively important points that should be not underestimated when creating my own game. Being able to socialize whilst playing a game may increase participation and fun. Students can also talk to each and discuss ideas about what is the correct way to go about solving a problem. 

%Fun
\subsubsection{Fun}
Games being fun is one of the main reasons that people want to make non-gaming concepts into games. If people will enjoy what they are doing more when they are playing games, then games are a viable solution for all sorts of media, including educational resources. A game being fun will capture an audience more and they will be more likely to play again because of the enjoyment they have got whilst playing it.

Research published by the UKIE stated that the UK games market is worth £3.944 billion in 2014 \cite{gamesresearch}. This much money would not have been spent on games if the consumer did not enjoy playing games. As mentioned by Rosemary Garris, Robert Ahlers and James E. Driskell, users make a clear cut choice to whether they are enjoying a game based on their own feelings  \cite{garris2002games}. 

%Hints and tips 
\subsubsection{Hints and Tips}
The final advantage of gamifying education that will be discussed is the ability to be able to give users hints and tips. Students learning a topic can often get stuck, and sometimes with nowhere to turn, may give up and try again another time. What game can offer to educational topics that more traditional methods can not offer is that they can give the player hints and tips at that moment whilst playing. If a player is struggling with a particular topic, a prompt could be the only motivation a student needs to carry on. This hint could help the student continue working through the problem instead of stopping. This can aid further learning and a longer time will be spent playing the game because the user will be able to complete the levels. 

Chen discussed this in his 2009 paper on giving users prompts to online learners \cite{chen2009effects}. He found that "the main factor affecting reflection levels is high levels prompts". He went on to say that "We believe that effective reflection can be achieved by reflection prompts". This means that prompts can be an effective tool for promoting learning performance through a process of reflection.

\subsection{Current Educational Games}

MyMaths is a popular subscription website that offers a number of mathematical puzzles, challenges and games for a student to work through. \cite{mymaths} From personal experience of using this platform when I was in secondary school, it was a much more exciting way of experiencing mathematics because it was more interactive than just completing questions, it was understanding mathematics in a completely different way. The key element to this platform being successful is that it is fun. It keeps students engaged by making a more fun way to learn.

A reason to me why MyMaths was a good interactive resource is that it was available anywhere there was a computer and internet connection, which are becoming more easily available resources these days. This means that the tasks on the website could be completed during dedicated lesson times, but also could be done during lunchtimes or at home. It was easily accessible and this will be one of the advantage of creating an online game.

A highly successful website that targets a wider audience to teach them how to code is Codecademy. \cite{codecademy} Codecademy is a website that provides the ability to learn the basics of certain programming languages. It does this by creating a variety of different levels with goals at every stage of the process. Some of the levels can contribute to the creation of a larger program. Using bite size chunks of code as levels is an idea that could be incorporated into this project. This splits the task into manageable pieces and makes it easier to understand.

Another concept that Codecademy uses is badges for completing levels, challenges and for having a streak of days where you complete challenges. Receiving badges makes the user gain a sense of achievement and will motivate them to continue with the website. Again, this would be something that could be considered for the project, as it encourages people to continue playing your game.

Codecademy makes levels progressively harder with harder coding concepts as the user goes through the levels. The user needs to be eased in to very basic concepts so they will continue playing the game through the levels and not give up. Another reason to introduce the basic concepts first is that it is common that basic concepts will need to be used in conjunction with the more complicated topics, which can then be introduced in later levels.

\section{Education}

Learning is a process that comes naturally throughout life. Throughout growing up, many skills are learnt without even realising we are learning them. When students are in education, they are consciously trying to learn new things to benefit themselves and to improve themselves. Learning could be defined as "A change in behaviour as a result of experience or practice" amongst other things. \cite{pritchard2013ways} Trying to learn things in the best way possible is our goal and there of many different ways of learning material. Similarly there are many different ways of teaching material. In this section some advantages of certain learning and teaching methods will be outlined. Gathering these different learning and teaching techniques will hopefully lead to the creation a more valuable game which can be used as an effective learning resource.

As summarised in Simulation and Gaming, games are perceived as more interesting than other ways of delivering material \cite{garris2002games}. In a study by Cohen, also mentioned in Simulation and Gaming, he found that 87\% of students were more interested in educational games than other classroom approaches. This highlights a want for educational games by students and an enthusiasm that might not be there for other approaches to teaching.

\subsection{Different learning styles}

Every person is different in which way they learn best and it would be rare that a person can only learn from one style, but the different ways of learning below are broadly the main ways people can take in and understand information.

\begin{itemize}
\item{Visual Learning}
\subitem{Visual learners like pictures and graphs. They'd prefer to see something written down in front of them rather than being said to them. Maps and other visual learning tools can also be effective for these kind of learners.}
\item{Auditory Learners}
\subitem{Auditory learners learn best by listening.They can take in information well from lectures and reading out loud to themselves. Good techniques for Auditory learners would be to record themselves saying information or hearing the information being said by others.}
\item{Kinaesthetic Learners}
\subitem{Kinaesthetic learning learn best by doing stuff. They like to touch and feel objects. If information is presented to them they learn best by writing it down. Playing a game where they were actively playing it would suit them well.}
\end{itemize}

\cite{mcdonald2012different} \cite{montemayor2009learning}

%Not really lit review
From these different learning styles, we can already see that different people learn in different ways. Although Kinaesthetic learners may benefit the most from playing the game, aspects of other learning styles could be included in the game so that all different types of learners can benefit from the game. For Visual learners, I have an idea that a level summary that could be displayed after a certain number of levels. This could be a range of information summarising what techniques have been used by the player. If this information were available to save and print, then it may be of great benefit to a visual learner.

Auditory learners could also benefit from playing a game. An idea to help auditory learners learn effectively would be include audio clips within the game. This could be after the levels, in the post game summary. There could also be videos to accompany the game. This could make the game more interesting, but it could also have a increased impact with auditory learners if the audio content was relevant and educational.

\subsection{Current Natural Deduction educational resources}

Natural Deduction has been taught in schools and universities for a long period now. With the more recent creation of the internet, there is now a number of different resources online which teach people the basics of natural deduction. All of them teach it in slightly different ways and each method of teaching the material will benefit a different group of learners better. In my game I will try and grasp some of the better characteristics of these ways of learning material and wrap them all up within my game.

\begin{itemize}
\item{Notes}
\subitem{Notes for Natural Deduction are widely available online written by different groups of people. Enthusiasts and academics all have notes available which help for different levels of understanding Natural Deduction. One of the main advantages of notes are that they are clearly laid out and organised. Many sets of notes are written in LaTex which gives them a professional and good aesthetic appearance.\cite{intro} Notes will be effective for visual learners who can see things written down and learn from them. One of the main advantages that I would like to incorporate into my game would be having easily displayed information available to view and also to print off for reference. }
\item{Lectures}
\subitem{Lectures can be an invaluable resource for auditory learners. Academics in their field can really convey information effectively and explain it well because they are experts in their field. Lectures typically include a visual aid, which could be a slide show or writing on a blackboard to help students understand the concepts. \cite{natdeclecture}This will benefit visual learners too, as they can see what is being explained. The important thing about lectures that could be included in the game is the audio. Audio is important to develop understanding and to explain concepts in a different way to the visual material.}
\item{Videos}
\subitem{Online videos can be found online for teaching Natural Deduction, such on websites like YouTube. \cite{vid2013} These are more interactive than tutorials online which are just powerpoint slides because they have audio over the top of them explaining what is happening. This particular video, uploaded by the user PhilHelper on YouTube,  uses graphics and examples to clarify some of the mathematical points. The mathematical notation is written next to the example, which keeps the mathematical notation in the forefront of the viewer's mind. This will appeal to visual learners. He is also speaking through what is on screen, so will benefit auditory viewers also. Some of the materials that he has in his videos would be effective in the project between levels of the game, to reinforce the learning.}
\end{itemize}

From these common techniques of delivering material, there is a trend that visual and auditory learners are well catered for through Notes and Lectures, as well as Videos catering for both. Kinesthetic learners may miss out somewhat by these traditional ways of learning.


\subsection{Online Natural Deduction resources}

Whilst not strictly an educational resource, Automated Theorem Provers are computer driven systems that can automatically work out a proof tree from a given end point \cite{automated}.These have their advantages as they automatically create a proof, so once the theory has been learnt by an individual, it can be checked by the automated theorem prover. What this project wants to achieve would be to have an automated theorem prover which can guide a user through creating a proof for themselves. The game will have all of the logic ingrained in the code, but the objective is to make the user learn as they go along. 

One similar project that has been undertaken is a Proof Assistant for Natural Deduction logic \cite{gasquet2011panda}. This tool has many good points that are worth highlighting. The notation is in keeping for what will be included in this project. It is also a very interactive system that will definitely benefit kinaesthetic learners. Things that I would like to focus on in this project is the aspect of becoming a game. Utilising levelling systems, fun and giving hints and tips to benefit the user. Additional features of some sort of social aspect would also improve engagement in the game.

By creating this project, it aims to target this type of learner through the interactive process of getting the player to 'do' things. If Natural Deduction can be taught effectively as a game, all the different learning types will have some benefit from playing, but especially kinaesthetic learners.

\section{Natural Deduction}

Natural deduction is a mathematical logic system. It is a system for proving statements of formal logic. It is used to show if certain statements are valid and to prove them. \cite{BriefHistory} Natural Deduction has a typical mathematical notation system that could create a barrier between the majority of general public understand how Natural Deduction works. Through this project, I hope to break down this barriers and make Natural Deduction more accessible to a wider range of people. 

Natural Deduction is used in a way that a large problem that needs to be proved is broken down into singular steps which are easily manageable. Each individual step is proved, and then that information can be used going forward in the proof. \cite{LogicND} These easier to understand proofs can then contribute to a larger Natural Deduction proof which was not trivial at the start. This lends itself well to leveling up system in my game, where the user could complete easier tasks in early levels and end up joining all of their findings together to prove a more complex theorem. 

\subsection{History}

Natural Deduction as an idea formally came to be in 1934/35 when Gerhard Gentzen and Stanas\l{}aw Ja{\'s}kowski both produced papers independently of each other. \cite{jaskowski1934rules} \cite{1964} No content had been published on this topic before and this led to the start of what is now known as the Natural Deduction Logic system. Both papers published in the same year had a very similar topic area of taking basic logic as a given- assumed to be true- and then creating a method of exploring what can be done with these basic axioms to prove more complicated theorems.  \cite{BriefHistory}

Whilst this was the true start of where the ideas of Natural Deduction came from, it did not pick up widespread appeal until educational textbooks begun to publish this logical system in the 1950s.  \cite{BriefHistory} At this point the Natural Deduction language was shaped into what it was today. Features and extra bits were added to the language that were not previously available before and more and more people began to learn about the concepts of Natural Deduction logic. People were taught how to solve practical problems using logic that had discovered and taught to them.

Not only is Natural Deduction important for practical uses in solving proofs. Over the 1960's and even continuing now, Natural Deduction has been used as a theoretical tool, and lots of research has gone into it to improve how Natural Deduction is applied. The aim of this is to make practical Natural Deduction problems easier to solve and quicker, possibly using computing resources to do so. The way Natural Deduction proofs are used has expanded, partly due to the work of Prawitz and Raggio on Normal Proofs. \cite{philNadDed}  More work continues to be done on many streams of Natural Deduction today to improve the way people use the logic system.

\subsection{Why Natural Deduction is suitable to make a game for}

Natural Deduction is a logical proof system which is thought of as closely aligned to people's natural reasoning patterns. \cite{arthur2011natural} It should, in theory, be easy to teach a system that is closely related to people's intuitive reasoning patterns. In reality, it might be that the mathematical notation could be one of the main things standing in the way of people learning this logical system. By creating something that is easier to grasp and understand, users should be able to connect with their intuition once again and understand the concepts behind Natural Deduction.  

Because of its close connection with being taught since the 1950s, this way of teaching logic has been one of the primary methods for many years now. Because of its ongoing reputation as a primary way of teaching mathematical logic in both mathematics and philosophy, it is still very popular today. An easily accessible way to learn this new 'language' for the first time can only be beneficial in my opinion.

Natural Deduction is a core system of logic that has be used regularly over the last fifty years. This importance has been shown through the amount of educational resources produced and the amount of material that has be written on the topic of Natural Deduction. The lacking of a fun, easy way to learn Natural Deduction is a shame and is why I feel it is necessary to create a game based on this logical system.

This project provides academic merit because of the alternative way that Natural Deduction logic will be taught in. Whilst primarily benefiting users and students who use the software for their own educational benefit, many academic experts and educational leaders in logic fields should find this paper interesting due to different ways of conveying traditional logical information to people who are not familiar with the concepts. 

\subsection{Mathematics}

This section will cover the basics of mathematics that will be implemented into the project. Most textbooks use similar notation but Fitch notation written in lines is sometimes used. For the purpose of this project, tree-like presentations will be used. 

\subsubsection{Notation}
\begin{itemize}
\item{And (Conjunction) $\wedge$}
\subitem{A and B both have to be true for A$\wedge$B to be true.}
\item{Or (Disjunction) $\vee$}
\subitem{Either A has to be true or B has to be true or both for A$\vee$B to be true.}
\item{Implies $\Rightarrow$}
\subitem{The statement A$\Rightarrow$B says that whatever A is, it must also hold for B as well. This shows consequence. You cannot however deduce A from B with this notation.}
\item{For all $\forall$}
\subitem{All elements in a set make a statement true then the for all symbol can be used appropriately. $ S=\{1,2,3\}$ $\forall x \in S,$ $ x < 4$.}
\item{There exists $\exists$}
\subitem{This means that at least one item in a particular set exists and holds true. $ S=\{1,4,9\}$ $\exists x \in S,$ s.t $ x > 7$.}
\item{If and only if $\Leftrightarrow$}
\subitem{A$\Leftrightarrow$B means that we can deduce A from B and we can also deduce B from A.}
\item{Not $\lnot$}
\subitem{The opposite of being true. A being false implies $\lnot$A is true.}
\end{itemize}
\cite{intro}
\subsubsection{Rules}

Everything on the top line is assumed to be true, with each statement separated by a comma. What is deduced is below the line. In proofs, there will be multiple lines on top of each other.

\paragraph{Conjunction introduction}

Introducing the 'and' symbol into the proof means that given two things that are deemed to be true, we can conclude both together are true. In formal mathematical notation, it would be written as shown in equation 1 below.

\bigskip \centerline{\infer[\wedge I]{A \wedge B}{A & B}}
\bigskip
A real life example of this would go as follows:
\begin{enumerate}
\item{It is sunny.}
\item{It is hot.}
\item{Therefore it can be concluded: It is sunny and hot.}
\end{enumerate}

\paragraph{Conjunction elimination}

Whilst the introduction takes two pieces of separate pieces of information and makes it into one new piece of information, conjunction elimination takes one piece of already assumed information and from this we can deduce two separate pieces of information. Equations 2 and 3 show what can be achieved.

\bigskip \centerline{\infer[\wedge E1]{A}{A \wedge B}}
\bigskip
\centerline{\infer[\wedge E2]{B}{A \wedge B}}
\bigskip
Similarly to the previous example we use conjunction, but instead of deducing the conjunction, we already have it, so we can deduce the individual components.
\begin{enumerate}
\item{It is sunny and hot.}
\item{Therefore it can be concluded: It is sunny.}
\item{It can also be concluded: It is hot.}
\end{enumerate}

\paragraph{Implication introduction}
This takes two statements. The first of which is assumed to be true. Now if we can deduce from this another element is true then trivially one must imply the other. 

\bigskip \centerline{\infer[\rightarrow I /x]{A \rightarrow B}{[x:A] & B}}
\bigskip

\paragraph{Implication elimination (Modus ponens)}

In implication elimination we are given an implication and (at least) one other piece of information and from this we can eliminate the implication to conclude that another element is also true. In Equation 4 below, we see that A $\rightarrow$ B is true but this is not enough to conclude that A or B is individually true. By being given the extra information that A is true, we can then use this rule to deduce B must also be true. 

\bigskip \centerline{\infer[\rightarrow E]{B}{A \rightarrow B & A}}
\bigskip
A practical example of this rule being used may be clearer to understand:
\begin{enumerate}
\item{James eating chocolate causes James to be happy.}
\item{James is eating chocolate.}
\item{It can be concluded:  James is happy.}
\end{enumerate}

In relation to Equation 4 above, James eating chocolate is A and James being happy is B. When the fact is given that James \textit{is} eating chocolate, then it can be concluded that James is happy and this is true on its own because of the other information we have.

\paragraph{Disjunction introduction}

This introduces the 'or' symbol into an equation based on the fact that one of the variables is true. This is because only one of the variables need to hold true for the disjunction to hold true. This is shown in Equation 5 and then an example is given in text.

\bigskip \centerline{\infer[\vee I]{A \vee B}{A}}
\bigskip

\begin{enumerate}
\item{Football is a sport.}
\item{Football is a sport or A Lemon is a sport.}
\end{enumerate}

The second variable does not have to be true because the first variable, in this case 'Football', makes the statement hold true. The disjunction would still hold true if both the variables held true.

\paragraph{Disjunction elimination}

This rule is very slightly more complex and gives a flavour of what complexity is possible with natural deduction proofs. This is still a simple rule and this list is not exhaustive of rules. Disjunction elimination manages to remove the disjunction because of other information given. There are two implication statements, both implying the same thing. Both these implications hold true. There is then a third true statement, which is a disjunction between the first variables of each implication. Because of the disjunction, it holds that at least one of the variables is true, meaning that the third variable that is implied is true. We conclude that the implied variable is true. Displayed in Equation 6 is a much simpler way to explain the rule using formal mathematical notation.

\bigskip \centerline{\infer[\vee E] {C}{A \rightarrow C, B \rightarrow C, A \vee B}}
\bigskip

An even easier way to understand this is to give a simple example.
\begin{enumerate}
\item{James eating chocolate causes James to be happy.}
\item{James eating pizza causes James to be happy.}
\item{James is eating chocolate or James is eating pizza \textit{(or both)}.}
\item{It can be concluded: James is happy.}
\end{enumerate}

This is exactly the same to what is displayed above but using examples instead of complicated mathematical notation.

\cite{LogicND} \cite{intro} \cite{philNadDed}

\section{Concluding remarks}

The three sections in this literature review: Gamification, Education and Natural Deduction have all touched on areas that will be brought together in this project. 

The benefits of games and what makes people play them have been discussed. The advantages that a game could bring to an educational resource has also been researched in detail. Current games that are successful in the real world show what games can do if created in the correct way. I conclude that a game can be a valuable resource for educational purposes where people can learn better. This, in my opinion, justifies why a game should be made of Natural Deduction.

Education, the second section, talked about different learning styles. There is a lack of kinaesthetic leaning in mainstream Mathematics and this is exactly what a game can provide. I looked at similar resources and projects to my own and how I feel they can be improved upon. Due to the lack of similar, helpful resources there is definitely a need for this kind of online platform that can help educate.

Finally, an introduction to Natural Deduction. This gave a background to the history and to reasons why Natural Deduction should be used to make a game. Natural deduction is still one of the core logic concepts and it is important to make sure people know about it. 

All these things lead me to believe that a project in this area would be extremely worthwhile which will have value to many different groups. Academics and Students alike should find benefit in a resource of this kind. This project, whilst mainly appealing to Mathematicians due to the content, should also appeal to other key departments such as Computer Science where Natural deduction is used extensively. Philosophers should also find this project interesting due to the ongoing research into whether gaming is an effective teaching resource.    

\newpage
\bibliography{fyprefs}
\end{document}