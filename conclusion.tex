%\section{A Brief Summary}

This project has demonstrated how users can learn mathematics using a game. The learning tool created fuses together ideas from Mathematical logic and Gamification principles. The Mathematical theory is central to the project and throughout it encourages users of the software to learn. The aim of the game is to teach users about the Natural Deduction proof system.  Educational values were held important throughout the duration of this work and therefore it was necessary to ensure that Natural Deduction knowledge was absorbed through the medium of game. The author wanted to create software which offered something that was not currently available elsewhere and that users which had little knowledge of Natural Deduction proof systems found educational and fun.

%\section{Why this project has value}



The culmination of research studied in Chapter \ref{lit}, in topics of gamification and Mathematical education, has inspired the software that has been designed and implemented in this project. The inspiration has mainly come from how Kinaesthetic learning techniques can be effective in the learning process and how gaming principles can be integrated with education.

There is no current software available which offers the balance of learning and Natural Deduction Mathematics that this project offers. Some current Educational games are mentioned in Section \ref{lit:egames} and also Natural Deduction resources are looked at in Section \ref{natedu}. The software created in this project is unique as it takes aspects from both to create a brand new educational game. The game delivered has improved the educational Natural Deduction resources available and explored whether demand is there for a engaging Natural Deduction learning tool. The preliminary study suggests that the software can be effective at teaching users about the Natural Deduction proof system.

Software has successfully been created to a point where user testing has been done and the results analysed. A Javascript game is currently hosted online which users can make use of to learn about Natural Deduction proof systems. 

%\section{Conclusion of results}

The initial feedback from the results of the Alpha testing have been highly encouraging. User feedback suggests that the software achieves its main aim of being an effective Natural Deduction educational resource. In general they thought that game was a better resource to learn from than learning through traditional methods like classroom sessions.

Also found in the survey was that users felt they knew more about Natural Deduction after playing the game, giving encouraging signs that this sofware can be used as a valuable learning tool for students. 

Usability of the game received a mixed response. There were few comments on the drag and drop feature suggesting that this was effective but there was comment that some users stopped because of usability issues to do with the lack of ability to undo moves made.

One of the other key results is how fun the users found the game. This is important because the software is supposed to be a more enjoyable way to learn about Natural Deduction than current resources available. Users indicated that they were having an average amount of fun. I believe that an improvement in usability will have a positive effect on the amount of fun the user will have.   

 Overall I am very proud of the achievements I have accomplished during this project. I am proud of the software created and felt it was good enough to let users test it. I have found, in conjunction with the results, that the game created has a solid framework which only need minor tweaks to make it more effective as a Natural Deduction learning resource. This project has been a great learning experience both personally and in the topic that has been studied. In my opinion, the link between games and education will grow stronger as people grow up with electronic devices and so if educational tools can be effectively provided in the form of games, then they can have a great effect. Whilst this study is not broad enough to draw any research conclusions, I am confident that this software is suitable for enabling users to learn from it.


%The results discussed in Section \ref{ssection:edu} show that over 80\% of the users felt that they were learning more about Natural Deduction. This shows that the core principles of the game are correct and the groundwork has been laid for a capable software product. 84.6\% of users thought that playing this game is a better way to learn than in traditional methods. These were methods such as using pen and paper or learning in a classroom. Whilst the initial survey taken was small, this overwhelmingly positive response advocating the learning capabilities the game offers shows that small tweaks to the project can make it a successful and unique learning product.  

%The drag and drop of the game went down well. Little comment was made on the reliability of the drag and drop and the users mainly had mistakes down to user error. This provides a solid base to build from. There was also little comment about the mathematics. Whilst most users did not know sufficiently about Natural Deduction to comment, the mathematical foundations this game has been built on will lead to future and more complicated proofs to be put in place.

%100\% of users found the tutorial useful and two thirds found using the hints were beneficial. This shows a good support network has been put in place to help users throughout the game. Rewriting of the hints has been necessary due to not meeting Requirement \ref{ssection:fun7}. Improvement of hints was a short term objective to improve the level reached by the user in the game. These changes will lead to a more comprehensive package to assist users who are stuck within the game. The aim of this is to prevent them from giving up and complete more of the game where new concepts are introduced.

%Another reason that users stopped playing was because of usability issues. This meant some participants didn't enjoy the experience. Users found this game average fun, with a mean score of 2.93/5. Improvement can be made and users will find the game more fun by improving usability. If a better user interface can be put in place then this will give users a better experience. Usability was commented on frequently in the user testing and small improvements will lead to a large improvement in engagement and enjoyment of users.

%Future Developments for the Project and future research



%Final Paragraph


